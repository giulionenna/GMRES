\documentclass{article}
\usepackage[utf8]{inputenc}

\usepackage{geometry}
\usepackage{bm}
\geometry{a4paper}
\usepackage{latexsym}
%\usepackage[dvips]{graphicx}
\usepackage{epsfig}
\usepackage{amsmath}
\usepackage{amsfonts}
\usepackage{amssymb}
\usepackage{eucal}
\usepackage{mathrsfs}
\usepackage{wasysym}
\usepackage{setspace}
\usepackage{float}
\usepackage{color}
\usepackage{rotating}
\usepackage{stmaryrd}
\usepackage{lineno}

\numberwithin{equation}{section}
\frenchspacing
%%
\usepackage{amsthm}
\usepackage[]{mdframed}

%%%%INSERITI ADESSO%%%%
\usepackage{amsmath}
\usepackage{amsfonts}
\usepackage{amssymb}
\usepackage{amsthm}
\usepackage{mathrsfs}
\usepackage{eucal}  
\theoremstyle{definition}
\usepackage{accents}
\usepackage{array}
\usepackage{cases}
\usepackage{graphicx}
\usepackage{booktabs}
\usepackage{caption}
\usepackage{cancel}
\usepackage{bbm}
\usepackage{subfig}
\usepackage{enumitem}
\usepackage{movie15}
 \usepackage{algorithm}
\usepackage{algpseudocode}
\usepackage{tabularx}
\usepackage{longtable}
 
% Font Management
\usepackage[T1]{fontenc}       % 8 bit font encoding: includes all accents
\usepackage{bm}                % alternative to \bs provided by package amsmath
\usepackage{bbm}               % alternative to \mathbb;  usage: \mathbbm{}
%\usepackage[mathscr]{eucal}    % alternative to \mathcal; usage: \mathcal{}
\usepackage{color}             % for text in colour
\usepackage{verbatim}          % environment for commenting out blocks of text
%\usepackage{exscale}           % needed to scale cmdx fonts
%\usepackage{ae,aecompl}        % see http://www.ctan.org/tex-archive/fonts/ae
%%%%%%%%%%%%%%%%%%


\theoremstyle{plain}
\newtheorem{thm}{Theorem}[section]
\newtheorem{lem}[thm]{Lemma}
\newtheorem{prop}[thm]{Proposizione}
\newtheorem*{cor}{Corollario}

\theoremstyle{definition}
\newtheorem{defn}{Definizione}[section]
\newtheorem{conj}{Congettura}[section]
\newtheorem{exmp}{Esempio}[section]

\theoremstyle{remark}
\newtheorem*{rem}{Osservazione}
\newtheorem*{note}{Nota}

\DeclareMathOperator*{\argmin}{argmin}
\DeclareMathOperator*{\argmax}{argmax}

\newcommand{\dom}{\mathrm{dom}}
\newcommand{\im}{\mathrm{im}}
\newcommand{\sign}{\mathrm{sign}}
\newcommand{\abs}{\mathrm{abs}}
\newcommand{\e}{\mathrm{exp}}

\setlength{\textwidth}{15 cm}
\setlength{\textheight}{23.5 cm}



%%%%%%%%%%%%%%%%%%%%%%%%%%%%%%%%%%%%%%%%%%%%%%%%%%%

\usepackage[utf8]{inputenc}
\usepackage[T1]{fontenc}
\usepackage{lmodern}

\usepackage{hyperref}
\hypersetup{%
    pdfpagemode={UseOutlines},
    bookmarksopen,
    pdfstartview={FitH},
    colorlinks,
    linkcolor={blue},
    citecolor={blue},
    urlcolor={blue}
  }

%%%%%%% use PDFLATEX 

\usepackage{lipsum} %to insert random text

\usepackage{geometry} %for the margins
\newcommand\fillin[1][4cm]{\makebox[#1]{\dotfill}} %for the dotted line in the frontispiace

\usepackage{dcolumn}
\newcolumntype{d}{D{.}{.}{-1} } %to vetical align numbers in tables, along the decimal dot

\usepackage{amsmath}



%%%%%%% Local definitions
\newtheorem{osservazione}{Osservazione}% Standard LaTeX
\newtheorem{observation}{Observation}% Standard LaTeX

\newcommand{\BR}{\mathscr{B}_{\mathrm{R}}}
\newcommand{\T}[2]{T_{#2}#1}
\newcommand{\cT}[2]{T_{#2}^{*}#1}
\newcommand{\pder}[2]{\frac{\partial #1}{\partial #2}}

				 
%%%%%%%%%%%%%%%%%%%%%%%%%%%%%%%%%%%%%%%%%%%%%%%%%
%
% Inserito il codice Matlab
%
\usepackage{listings}
\usepackage{hyperref}
\usepackage{xcolor}
\lstset{
backgroundcolor=\color{white},   
basicstyle=\footnotesize\ttfamily,        
breakatwhitespace=true,         % sets if automatic breaks should only happen at whitespace
breaklines=true,                 % sets automatic line breaking
captionpos=b,                    % sets the caption-position to bottom
commentstyle=\color{gray},                
keepspaces=true,                 % keeps spaces in text, useful for keeping indentation of code
columns=flexible,
keywordstyle=\color{blue},       % keyword style
language=MATLAB,                 % the language of the code
morekeywords={*,...},            % if you want to add more keywords to the set
deletekeywords={gamma,beta},
numbers=left,                    % where to put the line-numbers; possible values are (none, left, right)
numbersep=10pt,                   % how far the line-numbers are from the code
numberstyle=\tiny\color{gray}, % the style that is used for the line-numbers
rulecolor=\color{black},         % if not set, the frame-color may be changed on line-breaks within not-black text (e.g. comments (green here))
showspaces=false,                % show spaces everywhere adding particular underscores; it overrides 'showstringspaces'
showstringspaces=false,          % underline spaces within strings only
showtabs=false,                  % show tabs within strings adding particular underscores
stepnumber=1,                    % the step between two line-numbers. If it's 1, each line will be numbered
tabsize=2,	                   % sets default tabsize to 2 spaces
frame=lines,                   %POSSO TOGLIERLO
} 



\title{Using GMRES in a Constraint Optimization Algorithm}
\author{Nenna Giulio, Ornella Elena Grassi}
\date{Computational Linear Algebra For Large Scale Problems}
\begin{document}
\maketitle
The aim of this homework is to implement the GMRES algorithm (Generalized minimal residual method) for solving linear systems and test its effectiveness in a quadratic constraint optimization problem. In particular, we will introduce an Interior Point Method (IPM) for solving a quadratic constraint optimization problem which is an iterative method that requires to solve two sparse large scale linear systems at each iteration. We will use our implementation of the GMRES algorithm to solve those linear systems and measure both the accuracy and the speed of the method compared to other linear solvers.

\section{Problem Setting}
The prototype of constrained quadratic programming problem that we are dealing with is the following:
\begin{linenomath}

	%\begin{subequations}
	\begin{align} 
		\min_{\bm{x}\in \mathbb{R}^{n}} \quad  &\frac{1}{2}\bm{x}^\mathrm{T}\mathbf{Q}\bm{x} + \bm{c}^{\mathrm{T}}\bm{x} \label{Sec1_Eq_prb_mat} \\
		\text{subject to}\quad & \mathbf{A}\bm{x} = \bm{b}, \nonumber\\
		& \bm{x} \geq \bm{0},\nonumber
	\end{align}  
	%\end{subequations}
\end{linenomath}
Where \(\mathbf{Q}\in \mathbb{R}^{n \times n}\) is a symmetric positive semi-definite matrix, \(\bm{c}\in \mathbb{R}^n\), \(\mathbf{A}\in \mathbb{R}^{K \times n}\) and \(\bm{b} \in \mathbb{R}^K\) define \(K\) equality constraints. Problem \ref{Sec1_Eq_prb_mat} can be reformulated without equality constraints if we consider that:
\begin{align}
  \mathbf{A}\bm{x} = \bm{b} \iff \begin{cases}
      \mathbf{A}\bm{x} \geq \bm{b},\\
      \mathbf{A}\bm{x} \leq \bm{b}.
  \end{cases}
\end{align}
Hence we can write a new problem, equivalent to \ref{Sec1_Eq_prb_mat}, in which there are only inequality constraints:
\begin{linenomath}
	%\begin{subequations}
	\begin{align}
		\min_{\bm{x}\in \mathbb{R}^{n}} \quad  &\frac{1}{2}\bm{x}^\mathrm{T}\mathbf{Q}\bm{x} + \bm{c}^{\mathrm{T}}\bm{x} \label{Sec2_Eq_yolo} \\
		\text{subject to}\quad & \hat{\mathbf{A}}\bm{x} \geq \hat{\bm{b}}, \nonumber
	\end{align}
	%\end{subequations}
\end{linenomath}
%
where $\hat{\mathbf{A}}\in \mathbb{R}^{(2K + n)\times n}$ and $\hat{\bm{b}}\in \mathbb{R}^{2K + n}$ are defined as
\begin{align}
	\hat{\mathbf{A}} = \begin{bmatrix}
	    \mathbf{A} \\
	    -\mathbf{A}\\
	    \mathbf{I}_{n\times n}
	\end{bmatrix}, \quad \quad  
	\hat{\bm{b}} = \begin{bmatrix}
	    \bm{b} \\
	    -\bm{b}\\
	    \bm{0}_{n}
	\end{bmatrix}.
	\label{Sec2_Eq_A_hat_B_hat}
\end{align}
We will solve the constraint quadratic programming problem as formulated in \ref{Sec2_Eq_yolo} since this formulation is valid both for problems with and without equality constraints.
\\
\\
We can now define the corresponding Lagrangian function as:
\begin{equation}
	\mathcal{L}(\bm{x}, \bm{\lambda}) := \frac{1}{2}\bm{x}^\mathrm{T}\mathbf{Q}\bm{x} + \bm{c}^\mathrm{T}\bm{x} + \hat{\bm{\lambda}}^\mathrm{T}(\hat{\mathbf{A}}\bm{x}-\hat{\bm{b}})
	\label{Sec1_Eq_lagrangian_B}
\end{equation}
where $\hat{\bm{\lambda}}\in\mathbb{R}^{2K+n}$ is the vector of Lagrangian multipliers associated with the inequality constraints and we can also write the corresponding KKT conditions:
\begin{equation} 
    \begin{cases} 
		\mathbf{Q}\bm{x} + \bm{c} - \hat{\mathbf{A}}^\mathrm{T}\hat{\bm{\lambda}} = \bm{0} &\text{Stationarity condition,} \\
		\hat{\mathbf{A}}\bm{x} - \hat{\bm{b}} \geq \bm{0} &\text{Primal feasibility,}\\
		\hat{\bm{\lambda}} \geq \bm{0} &\text{Dual feasibility,}\\
		\hat{\bm{\lambda}}^\mathrm{T}\left( \hat{\mathbf{A}}\bm{x} - \hat{\bm{b}}\right) = \bm{0} &\text{Complementary slackness condition.}
	\end{cases}
	\label{Sec1_Eq_kkt_before}
\end{equation}
We can manipulate the system \eqref{Sec1_Eq_kkt_before} in order to simplify the system of equations. In particular, we define a \textit{slack variable} $\bm{y} := \hat{\mathbf{A}}\bm{x} - \hat{\bm{b}} \in \mathbb{R}^{2K+n}$ such that the KKT conditions becomes
\begin{equation} 
    \begin{cases} 
		\mathbf{Q}\bm{x} + \bm{c} - \hat{\mathbf{A}}^\mathrm{T}\hat{\bm{\lambda}} = \bm{0} &\text{Stationarity condition,} \\
		\bm{y} \geq \bm{0} &\text{Primal feasibility,}\\
		\hat{\bm{\lambda}} \geq \bm{0} &\text{Dual feasibility,}\\
		\hat{\mathbf{A}}\bm{x} - \hat{\bm{b}} -\bm{y} = \bm{0} &\text{Slack variable constraint,}\\
		\hat{\bm{\lambda}}^\mathrm{T}\bm{y} = \bm{0} &\text{Complementary slackness condition.}
	\end{cases}
	\label{Sec1_Eq_kkt_after}
\end{equation}
\section{Predictor-Corrector IPM}
Let's implement a Predictor-Corrector Interior Point Method for finding the solutions of the system \eqref{Sec1_Eq_kkt_after}. We can rewrite the problem as finding the zeros of the function \(F\,: \mathbb{R}^{4K + 3n} \to \mathbb{R}^{4K + 3n}\)
\begin{equation}
    F(\bm{x}, \bm{y} ,\bm{\lambda}) = \begin{bmatrix}
                        \mathbf{Q}\bm{x} + \bm{c} - \hat{\mathbf{A}}^\mathrm{T}\hat{\bm{\lambda}}\\
                        \hat{\mathbf{A}}\bm{x} - \hat{\bm{b}} - \bm{y}\\
                        \mathbf{Y}\hat{\mathbf{\Lambda}}\bm{e}
                    \end{bmatrix} = \bm{0}
                    \label{Sec2_Eq_F}
    \end{equation}
    with the inequality constraints $\bm{y} = \hat{\mathbf{A}}\bm{x} - \hat{\bm{b}} \geq \bm{0}$ and $\hat{\bm{\lambda}} \geq \bm{0}$, where we have defined
    \[\mathbf{Y} = 	\begin{bmatrix}
            y_1 & & \bm{0}\\
            & \ddots & \\
            \bm{0} && y_n
            \end{bmatrix} \in\mathbb{R}^{(2K+n) \times (2K+n) }, \quad  
            \hat{\mathbf{\Lambda}} = 	
            \begin{bmatrix}
            \hat{\lambda}_1 && \bm{0}\\
            & \ddots & \\
            \bm{0} & & \hat{\lambda}_n \\
            \end{bmatrix} \in\mathbb{R}^{(2K+n)\times (2K+n)},\]
    and $\bm{e} = [1, \dots, 1]^\mathrm{T} \in \mathbb{R}^{2K+n}$. 
\\
\\
\noindent The main idea about \textbf{Interior Point Methods} is that, given a starting point \((\bf{x}_0, \bf{y}_0, \bf{\lambda}_0)\), we want to stay away from the boundary of the feasible set for as many iterations as possible since getting stuck on the boundary in early iterations kills the convergence speed of the algorithm. What we want instead is for the solution to gently approach the boundary in order to maximize the convergence speed. \\
This is accomplished in two main steps: the \textit{Predictor} step and the \textit{Corrector} step. In the Prediction step, given a current feasible solution \((\bf{x}_k, \bf{y}_k, \bf{\lambda}_k)\), we compute a simple \textit{Newton Step} to solve eq. \ref{Sec2_Eq_F}. We then estimate how far away we are from the feasible set and compute the \textit{Corrector} step which is simply a correction of the Newton step computed above needed to keep the boundary of the feasible set at a desired distance that will decrease during iterations.
\\
\\
\noindent Since we will solve Equation \eqref{Sec2_Eq_F} with the Newton method, we will need the Jacobian of \(F\), which is computed as
    \[ \mathbf{J}_F = \begin{bmatrix}
        \mathbf{Q} & \mathbf{0} & -\hat{\mathbf{A}}^\mathrm{T}\\
        \hat{\mathbf{A}} & -\mathbf{I} & \mathbf{0} \\
        \mathbf{0} & \hat{\mathbf{\Lambda}} &\mathbf{Y}
        \end{bmatrix}.\]
    The resulting Predictor-Corrector IPM algorithm is the following:
    \begin{itemize}
        \item We start from an initial point $(\bm{x}_0,\bm{y}_{0}, \hat{\bm{\lambda}}_{0})$ with the only request that $\bm{y}_0$ and $\hat{\bm{\lambda}}_0$ satisfy the primal and dual strict feasibility conditions, i.e. $\bm{y}_0, \hat{\bm{\lambda}}_0 > 0$, and we choose $\bm{x}_{0}$ which solves the condition $\hat{\mathbf{A}}\bm{x}_{0} - \hat{\bm{b}} -\bm{y}_{0} = \bm{0}$ in the \textit{least-square sense}.
        \item (\textbf{Predictor}) At step $k$, using the current three iterates $(\bm{x}_k, \bm{y}_k, \hat{\bm{\lambda}}_k)$, we compute the affine scaling step $(\Delta\bm{x}_k^\mathrm{aff}, \Delta{\bm{y}}_k^\mathrm{aff},  \Delta\hat{\bm{\lambda}}_k^\mathrm{aff})$ using the following Newton step
        \begin{equation} \label{Sec_2_Eq_NewtonStep1}
        \begin{bmatrix}
            \mathbf{Q} & \mathbf{0} & -\hat{\mathbf{A}}^\mathrm{T}\\
            \hat{\mathbf{A}} & -\mathbf{I} & \mathbf{0} \\
            \mathbf{0} & \hat{\mathbf{\Lambda}}_k &\mathbf{Y}_k
            \end{bmatrix} 
        \begin{bmatrix}
            \Delta\bm{x}_k^\mathrm{aff} \\  \Delta\bm{y}_k^\mathrm{aff} \\
            \Delta\hat{\bm{\lambda}}_k^\mathrm{aff}
        \end{bmatrix} =
        -\begin{bmatrix}
            \mathbf{Q}\bm{x}_k + \bm{c} - \hat{\mathbf{A}}^\mathrm{T}\hat{\bm{\lambda}}_k\\
            \hat{\mathbf{A}}\bm{x}_k - \hat{\bm{b}} - \bm{y}_k\\
            \mathbf{Y}_k\hat{\mathbf{\Lambda}}_k\bm{e}
        \end{bmatrix}
    \end{equation}
    %
    where, again, we've defined $\mathbf{Y}_k = \mathrm{diag}\left( (\bm{y}_k)_1, \dots, (\bm{y}_k)_n\right)$ and $\hat{\mathbf{\Lambda}}_k = \mathrm{diag}( (\hat{\bm{\lambda}}_k)_1, \dots, (\hat{\bm{\lambda}}_k)_n)$.
    \item We then compute the affine step-length $\alpha_{k}^\mathrm{aff}>0$ in order to remain in the internal part of the feasible set by ensuring that $\bm{y}_k + \alpha_{k}^\mathrm{aff}\Delta\bm{y}_k^\mathrm{aff}>0$ and $\hat{\bm{\lambda}}_k + \alpha_{k}^\mathrm{aff}\Delta\hat{\bm{\lambda}}_k^\mathrm{aff}>0$. This is accomplished by setting the step-length as
    \begin{align}
        \alpha_{k}^\mathrm{aff} = \min\left\{ 1, \, \min\left\{ -\frac{(\bm{y}_k)_i}{(\Delta\bm{y}_k^\mathrm{aff})_i} \, : \, i =1, \dots, 2K + n,\,\,\, (\Delta\bm{y}_k^\mathrm{aff})_i < 0\right\}\right.\nonumber\\
        \min\left.\left\{ -\frac{(\hat{\bm{\lambda}}_k)_i}{(\Delta\hat{\bm{\lambda}}_k^\mathrm{aff})_i} \, : \,  i =1, \dots, 2K + n,\,\,\, (\Delta\hat{\bm{\lambda}}_k^\mathrm{aff})_i < 0\right\}    \right\}
    \end{align}
    \item We compute the affine complementarity measure $\mu_{k}^\mathrm{aff}$ and the complementarity parameter $\sigma_k$ as
    %
    \begin{linenomath}
        %\begin{subequations}
        \begin{align}
            \mu_{k}^\mathrm{aff} &= \frac{1}{n}\left(\bm{y}_k + \alpha_{k}^\mathrm{aff}\Delta\bm{y}_k^\mathrm{aff} \right)^\mathrm{T}\left(\hat{\bm{\lambda}}_k + \alpha_{k}^\mathrm{aff}\Delta\hat{\bm{\lambda}}_k^\mathrm{aff} \right)\\
            \sigma_k &= \left(\frac{\mu_{k}^\mathrm{aff}}{\mu_k}\right)^3,\quad\mu_k = \frac{\bm{y}_k^\mathrm{T}\hat{\bm{\lambda}}_k}{n}
        \end{align}
        %\end{subequations}
    \end{linenomath}
    Those are the measures that tell us "how far" we are from the boundary of the feasible set.
    \item (\textbf{Corrector}) We compute the affine corrector step $(\Delta\bm{x}_k, \Delta\bm{\lambda}_k,\Delta\bm{s}_k)$ using the following Newton step
    \begin{equation} 
    \begin{bmatrix}
            \mathbf{Q} & \mathbf{0} & -\hat{\mathbf{A}}^\mathrm{T}\\
            \hat{\mathbf{A}} & -\mathbf{I} & \mathbf{0} \\
            \mathbf{0} & \hat{\mathbf{\Lambda}}_k &\mathbf{Y}_k
        \end{bmatrix} 
        \begin{bmatrix}
            \Delta\bm{x}_k \\ 
            \Delta\bm{y}_k \\
            \Delta\hat{\bm{\lambda}}_k
        \end{bmatrix} =
        -\begin{bmatrix}
            \mathbf{Q}\bm{x}_k + \bm{c} - \hat{\mathbf{A}}^\mathrm{T}\hat{\bm{\lambda}}_k\\
            \hat{\mathbf{A}}\bm{x}_k - \hat{\bm{b}} - \bm{y}_k\\
            \mathbf{Y}_k\hat{\mathbf{\Lambda}}_k\bm{e}
        \end{bmatrix}+ 
        \begin{bmatrix}
            \bm{0} \\
            \bm{0} \\
            -\Delta\mathbf{Y}_{k}^\mathrm{aff}\Delta\hat{\mathbf{\Lambda}}_{k}^\mathrm{aff}\bm{e} + \sigma_{k}\mu_{k}\bm{e}
        \end{bmatrix}
        \label{Sec_2_Eq_kekmanet}
    \end{equation}
    %
    where $\Delta\mathbf{Y}_k = \mathrm{diag}\left( (\Delta\bm{y}_k)_1, \dots, (\Delta\bm{y}_k)_n\right)$ and $\Delta\hat{\mathbf{\Lambda}}_k = \mathrm{diag}( (\Delta\hat{\bm{\lambda}}_k)_1, \dots, (\Delta\hat{\bm{\lambda}}_k)_n)$.
    \item We then compute the step-length  $\alpha_{k}>0$ as before
    \begin{align}
        \alpha_{k} = \min\left\{ 1, \, \min\left\{ -\frac{\tau_k(\bm{y}_k)_i}{(\Delta\bm{y}_k)_i} \, : \, i =1, \dots, n,\,\,\, (\Delta\bm{y}_k)_i < 0\right\}\right.\nonumber\\
        \min\left.\left\{ -\frac{\tau_k(\hat{\bm{\lambda}}_k)_i}{(\Delta\hat{\bm{\lambda}}_k)_i} \, : \,  i =1, \dots, n,\,\,\, (\Delta\hat{\bm{\lambda}}_k)_i < 0\right\}    \right\}
        \label{Sec3_eq_tau}
    \end{align}
    where $\tau_k\in(0,1)$ controls how far we back off from the maximum step for which the inequality constraints are satisfied, i.e. $\bm{y}_k + \alpha_{k}\Delta\bm{y}_k\geq(1-\tau_k)\bm{y}_k$ and  $\hat{\bm{\lambda}}_k + \alpha_{k}\Delta\hat{\bm{\lambda}}_k\geq(1-\tau_k)\hat{\bm{\lambda}}_k$.
    \item (\textbf{Update}) We then update the values for the next iteration:
    \begin{align}
    \begin{cases}
        \bm{x}_{k+1} = \bm{x}_{k} + \alpha_{k}\Delta\bm{x}_k,\\
        \bm{y}_{k+1} = \bm{y}_{k} + \alpha_{k}\Delta\bm{y}_k,\\
        \hat{\bm{\lambda}}_{k+1} = \hat{\bm{\lambda}}_{k} + \alpha_{k}\Delta\hat{\bm{\lambda}}_k.
    \end{cases}\label{Sec3_Eq_sugoma}
    \end{align}
\end{itemize}

\subsection{Solution of the linear system}
At each step of the algorithm, we have to solve the linear system
\begin{equation} \label{Sec_2_Eq_NewtonStep}
    \begin{bmatrix}
        \mathbf{Q} & \mathbf{0} & -\hat{\mathbf{A}}^\mathrm{T}\\
        \hat{\mathbf{A}} & -\mathbf{I} & \mathbf{0} \\
        \mathbf{0} & \hat{\mathbf{\Lambda}}_k &\mathbf{Y}_k
    \end{bmatrix} 
    \begin{bmatrix}
        \Delta \bm{x} \\ \Delta \bm{y} \\ \Delta \hat{\bm{\lambda}} 
    \end{bmatrix} =
    \begin{bmatrix}
        \bm{r}_1 \\ \bm{r}_2 \\ \bm{r}_3
    \end{bmatrix}.
\end{equation}
%
Since the dimension of this linear system is very large, we need to split the equations in order to make the problem more manageable. From the third set of equations we get
\begin{equation} \label{Sec_2_eq_pio}
    \Delta \bm{y} = \hat{\mathbf{\Lambda}}^{-1} \bm{r}_3 -  \mathbf{D}^{-1}\Delta \hat{\bm{\lambda}}.
\end{equation}
where $\mathbf{D} = \mathbf{Y}^{-1}\hat{\mathbf{\Lambda}}$.
So, substituting \ref{Sec_2_eq_pio} into \ref{Sec_2_Eq_NewtonStep}, we obtain the following augmented system:
\begin{equation} \tag{Aug}\label{sec2_eq_augmented}
    \begin{bmatrix}
            \mathbf{Q} & -\hat{\mathbf{A}}^\mathrm{T} \\
            \hat{\mathbf{A}} & \mathbf{D}^{-1}
        \end{bmatrix}
        \begin{bmatrix}
            \Delta \bm{x}\\ \Delta \hat{\bm{\lambda}}
        \end{bmatrix} = 
        \begin{bmatrix}
            \bm{r}_1 \\ \bm{r}_2 + \hat{\mathbf{\Lambda}}^{-1}\bm{r}_3
        \end{bmatrix}.
\end{equation}
Moreover, we can exploit the new formulation to solve the system just by substitution. Indeed, if we apply $\mathbf{D}$ to the second equation, we get 
\begin{equation}
    \Delta \hat{\bm{\lambda}} = \mathbf{D}\left( \bm{r}_2 + \hat{\mathbf{\Lambda}}^{-1}\bm{r}_3 - \hat{\mathbf{A}}\Delta \bm{x} \right), \label{Sec3_Eq_antisemitismo}
\end{equation}
and, if we substitute back in the first equation of \eqref{sec2_eq_augmented}, we obtain the final system
\begin{align}
    \Delta \bm{x} &= \left( \mathbf{Q} + \hat{\mathbf{A}}^\mathrm{T}\mathbf{D}\hat{\mathbf{A}}\right)^{-1}\left( \bm{r}_1 + \hat{\mathbf{A}}^\mathrm{T}\mathbf{D}\left( \bm{r}_2 + \hat{\mathbf{\Lambda}}^{-1}\bm{r}_3 \right)\right) \nonumber\\
    &= \left( \mathbf{Q} + \hat{\mathbf{A}}^\mathrm{T}\mathbf{D}\hat{\mathbf{A}}\right)^{-1}\left( \bm{r}_1 + \hat{\mathbf{A}}^\mathrm{T}\left( \mathbf{D}\bm{r}_2 + \mathbf{Y}^{-1}\bm{r}_3 \right)\right). \label{Sec3_eq_byandlarge}
\end{align}
In \eqref{Sec3_eq_byandlarge} the matrix of the linear system is symmetric and positive semi-definite since $\mathbf{Q}$ and $\mathbf{D}$ are symmetric and positive semi-definite \footnote{$\mathbf{D} = \mathbf{Y}^{-1}\hat{\mathbf{\Lambda}}$ is indeed SPD because, by the primal and dual feasibility conditions, $y_k > 0$ and $\hat{\lambda}_k > 0$.}, so we will use the \texttt{pcg} solver.
To summarize, this approach solves the system by solving for $\Delta \bm{x}$ and then by substitution in the other two equations as presented in the following system:
\begin{equation}
    \begin{cases}
        \Delta \bm{x} &= \left( \mathbf{Q} + \hat{\mathbf{A}}^\mathrm{T}\mathbf{D}\hat{\mathbf{A}}\right)^{-1}\left( \bm{r}_1 + \hat{\mathbf{A}}^\mathrm{T}\left( \mathbf{D}\bm{r}_2 + \mathbf{Y}^{-1}\bm{r}_3 \right)\right),\\
        \Delta \hat{\bm{\lambda}} &= \mathbf{D}\left( \bm{r}_2 + \hat{\mathbf{\Lambda}}^{-1}\bm{r}_3 - \hat{\mathbf{A}}\Delta \bm{x} \right),\\
        \Delta \bm{y} &= \hat{\mathbf{\Lambda}}^{-1} \bm{r}_3 -  \mathbf{D}^{-1}\Delta \hat{\bm{\lambda}}.
    \end{cases}
    \tag{Sus}
    \label{Sec2_Ee_SUS}
\end{equation}



\end{document}